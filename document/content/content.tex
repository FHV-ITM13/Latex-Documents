% todo extract counter to main under \document
\setcounter{chapter}{6}

\chapter{$(\mu/\mu_I,10)-\sigma SA-ES$ am Kugelmodell}

In diesem Beispiel wird der Einfluss des $\mu$ in dem Algorithmus betrachtet. Hierfür wird die F-Dynamik in der Abbildung \ref{fig:plot_ue7_1} dargestellt. Als Testfunktion wird das Kugelmodell mit $N=100$ und $\mu=1...10$.

\img{0.8}{document/graphics/ue07_f_dynamic_octave.PNG}{F-Dynamik $(\mu/\mu_I,10)-\sigma SA-ES$ am Kugelmodell}{plot_ue7_1}

Die Abbildung zeigt, dass bei $\mu=10$ das Verfahren nicht mehr Terminiert. Begrenzt wird die Linie durch die Iterationsgrenze von 20.000.

\chapter{$(3/3_I,10)-\sigma SA-ES$ am Kugelmodell}

Diese Aufgabe betrachtet den Einfluss von T auf den $(3/3_I,10)-\sigma SA-ES$ Algorithmus am Kugelmodell mit $N=100$, $\sigma=0.1$ und $y_{init}=(1,...,1)^T$. Auch in dieser Aufgabe wird die F-Dynamik in der Abbildung \ref{fig:plot_ue8_1} dargestellt.

\img{0.8}{document/graphics/ue08_f_dynamic.PNG}{F-Dynamik $(3/3_I,10)-\sigma SA-ES$ am Kugelmodell}{plot_ue8_1}

Bei $\tau=0$ kann beobachtet werden, dass aufgrund der fehlenden $\sigma - Adaption$, der Wert konvergiert.

\img{0.8}{document/graphics/ue08_s_dynamic.PNG}{$\sigma$-Dynamik $(3/3_I,10)-\sigma SA-ES$ am Kugelmodell}{plot_ue8_2}

Auch die Abbildung \ref{fig:plot_ue8_2} zeigt, dass bei $\tau=0$ der Wert stagniert.

\chapter{Performance Vergleich}

Verglichen wird der Algorithmus $(1+1)-ES$ mit $1/5$ Regel mit dem Algorithmus $(3/3_I,10)-\sigma SA-ES$. Als Testfunktion wird das Kugelmodell mit $N=100$ verwendet. Der Vergleich erfolgt auf Basis der Generationen und aufgrund der Anzahl an Funktionsauswertungen.\\

\begin{table}[h]
  \begin{center}
  \begin{tabular}{|l|p{5cm}|p{5cm}|}
    \hline
     & $(1+1)-ES$ & $(3/3_I,10)-\sigma SA-ES$ \\
    \hline
    Generationen & 7301 & 986 \\
    \hline
    Funktionsauswertungen & 7301 & 10846 \\
    \hline
  \end{tabular}
  \caption{Ergebnisse der Ausführung}
  \end{center}
\end{table}

Aus diesen Ergebnissen kann ermittelt werden, dass die Dauer der Funktionsauswertung für die Auswahl des Algorithmus ausschlaggebend ist. Ist diese Dauer klein, ist der $(3/3_I,10)-\sigma SA-ES$ Algorithmus insgesamt schneller, da weniger generationen benötigt werden. Ist die Dauer der Funktionsauswertung hoch, ist der $(1+1)-ES$ Algorithmus aufgrund der kleineren Zahl der Funktionsauswertungen schneller.

\chapter{$(\mu/\mu_I,20)-\sigma SA-ES$ am verrauschen Kugelmodell}

Diese Aufgabe beschäftigt sich mit der Performance des Algorithmus $(\mu/\mu_I,20)-\sigma SA-ES$ am verrauschten Kugelmodell. Hierfür wird $F(y)=||y||^2+\mathcal{N}(0,1)$, $N=100$, $y_{init}=(10,...,10)^T$ und $\sigma_{init}=1$.
Der Algorithmus wurde über 3000 Generationen durchgeführt. In der Abbildung \todo{ref: Aufgabe 10} wird die Dynamik des Restzielabstandes $||y^{(g)}||$ des elterlichen Restkombinanten dargestellt.

\todo{Abbildung (Aufgabe 10): Dynamik des Restzielabstandes $||y^{(g)}||$}

\todo{Aufgabe 10}

\chapter{$(\mu/\mu_I,\lambda)-\sigma SA-ES$ an der Rastrigin-Funktion}

In dieser Aufgabe wird das Verhalten des $(\mu/\mu_I,\lambda)-\sigma SA-ES$ Algorithmus an der Rastrigin-Funktion (Abbildung \ref{fig:rastrigin}) untersucht.

\begin{figure}[h]
  \label{fig:rastrigin}
  \begin{center}
  
    $F(y)=\sum_{i=1}^N(a-a*cos(2\pi y_i)+y_i^2), a=2$
    
    \caption{Rastrigin Funktion}
  \end{center}
\end{figure}

Als initialwerte wurde $N=30$, $\sigma_{init}=1$ und $y_{init}={10,...,10}^T$ verwendet.

\todo{Aufgabe 11}

\chapter{CMSA-ES für Linsenoptimierung}

Diese Aufgabe soll zeigen, dass die Kovarianzmatrix-Adaption wichtig für die perfomante Optimierung des ES ist. Diese Annahme bestätigt die Abbildung \todo{ref: Aufgabe 12}. Diese Abbildung zeigt die Dynamiken des CMSA-ES mit dem $(\mu/\mu_I,\lambda)-\sigma SA-ES$ am Linsenproblem.

Als initialwerte wurde $N=30$, $\sigma_{init}=1$ und $y_{init}={10,...,10}^T$ verwendet.

\todo{Abbildung (Aufgabe 12): Dynamiken des CMSA-ES mit dem $(\mu/\mu_I,\lambda)-\sigma SA-ES$ am Linsenproblem}

\todo{Aufgabe 12}





